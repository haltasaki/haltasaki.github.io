%%%%%%%%%%%%%%%%%%%%%%%%%%%%%%%%%%%%%%
%%%%%%%%%%%%%%%%%%%%%%%%%%%%%%%%%%%%%%
%%%%%%%%%%%%%%%%%%%%%%%%%%%%%%%%%%%%%%
\documentclass[12pt]{jarticle}
\usepackage{epsfig,amsmath,amsfonts,amssymb,makeidx,ifthen,upgreek}
%%%%%%%%%%%%%%%%%%%%%%%%%%%%%%%%%%%%%%
%%%%%%%%%%%%%%%%%%%%%%%%%%%%%%%%%%%%%%
%%%%%%%%%%%%%%%%%%%%%%%%%%%%%%%%%%%%%%
%%%%%%%%%%%%%%%%%%%%%%%%%%%%%%%%%%%%%%%%%%%
\setlength{\oddsidemargin}{0.0truecm}
\setlength{\textwidth}{16.5truecm}
\setlength{\textheight}{20truecm}
\setlength{\topmargin}{0.0truecm}
%%%%%%%%%%%%%%%%%%%%%%%%%%%%%%%%%%%%%%%%%%%
%%%%%%%%%%%%%%%%%%%%%%%%%%%%%%%%%%%%%%%%%%%
\newcommand{\tolabel}[2]{\mathop{\longrightarrow}^{\rm #1}_{\rm(#2)}}
\newcommand{\too}[1]{\stackrel{\rm #1}{\longrightarrow}}
\newcommand{\both}[1]{\stackrel{\rm #1}{\longleftarrow\hspace{-4mm}\rightarrow}}
%The following did not work in in-line equations.
%\newcommand{\too}[1]{\mathop{\longrightarrow}^{\rm #1}}
%\newcommand{\both}[1]{\mathop{\longleftarrow\hspace{-4mm}\rightarrow}^{\rm #1}}
%%%%%%%%%%%%%%%%%%%%%%%%%%%%%%%%%%%%%%%%%%%
\newcommand{\tooa}{\too{a}}
\newcommand{\bothaq}{\both{aq}}
\newcommand{\tooaq}{\too{aq}}
\newcommand{\tooi}{\too{i}}
\newcommand{\tooiq}{\too{iq}}
\newcommand{\tooig}{\too{i'}}
\newcommand{\bothiq}{\both{iq}}
\newcommand{\pv}{p_{\rm v}}
\newcommand{\pc}{p_{\rm c}}
\newcommand{\Cv}{C_{\rm v}}
\newcommand{\Cp}{C_{\rm p}}
\newcommand{\sumim}{\sum_{i=1}^m}
\newcommand{\xmin}{x_{\rm min}}
\newcommand{\xmax}{x_{\rm max}}
\newcommand{\amin}{\alpha_{\rm min}}
\newcommand{\amax}{\alpha_{\rm max}}
\newcommand{\VL}{V_{\rm L}}
\newcommand{\VG}{V_{\rm G}}
\newcommand{\vso}{v_{\rm S}}
\newcommand{\vl}{v_{\rm L}}
\newcommand{\vg}{v_{\rm G}}
\newcommand{\THi}{T_{\rm H}}
\newcommand{\TL}{T_{\rm L}}
\newcommand{\dn}{\searrow}
\newcommand{\DT}{\Delta T}
\newcommand{\Dp}{\Delta p}
\begin{document}
%%%%%%%%%%%%%%%%%%%%%%%%%%
%%%%%%%%%%%%%%%%%%%%%%%%%%
\begin{flushright}
\footnotesize
「熱力学 --- 現代的な視点から」への訂正(2010年3月1日)
\end{flushright}
%%%%%%%%%%%%%%%%%%%%%%%%%%
%%%%%%%%%%%%%%%%%%%%%%%%%%
\noindent
p.151, 問題の修正 
\bigskip

\noindent
{\bf 7.9}(7-7節)物質量$N$の純物質を用意し、
\begin{equation}
(T';V'_0,N)\tooiq(T';V'_1,N)\tooaq(T;V_1,N)\tooiq(T;V_0,N)\tooaq(T';V'_0,N)
\tag{7.70}
\end{equation}
というCarnotサイクルを考える。
サイクルのあいだ系はつねに気体と液体が相共存した状態にある。
考えやすいよう$V'_0<V'_1<V_1>V_0>V_0'$および$T'>T$としよう。
このサイクルにCarnotの定理を適用することでClapeyronの関係を示せ。
$\DT=T'-T$が微小としてエンタルピーの変化に着目するとよい。


\bigskip\bigskip
\noindent
p.151, 解答の修正 
\bigskip

\noindent
{\bf 7.9} サイクル(7.70)に現れる状態に順に$A$, $B$, $C$, $D$と名前をつける。
気液が共存しているので、圧力は、$C$と$D$の間でつねに$\pv(T)$、$A$と$B$の間でつねに$\pv(T+\DT)=\pv(T)+\Dp$である。
一般に、微小な断熱準静操作 $(T;V,N)\tooaq(T+\Delta T;V+\Delta V,N)$ での
エンタルピー $H=U+pV$ の変化は(エネルギー保存則 $\Delta U+p\Delta V=0$より)
$\Delta H=\Delta U+p\Delta V+V\Delta p=V\Delta p$ である。
よって$H(B)-H(C)=V_1\Dp$, $H(A)-H(D)=V_0\Dp$がわかる。
等温準静操作での発熱量はエンタルピーの差で書けるので、$D$から$C$に移る際に気化する物質の量を$N'$とすると、$Q(D\to C)=H(C)-H(D)=H_{\rm vap}(T;N')$である。
一方、上の考察から$Q(A \to B)=H(B)-H(A)=Q(D\to C)+(V_1-V_0)\Dp$である。
$Q(A\to B)/Q(D\to C)=1+(\DT/T)$に得られた表式を代入し、$(V_1-V_0)/N'=\vg(T)-\vl(T)$に注意すればClapeyronの関係が得られる。


%%%%%%%%%%%%%%%%%%%%%%%%%%%%%%%%%%%%%%
%%%%%%%%%%%%%%%%%%%%%%%%%%%%%%%%%%%%%%
%%%%%%%%%%%%%%%%%%%%%%%%%%%%%%%%%%%%%%
\end{document}
%%%%%%%%%%%%%%%%%%%%%%%%%%%%%%%%%%%%%%
%%%%%%%%%%%%%%%%%%%%%%%%%%%%%%%%%%%%%%
%%%%%%%%%%%%%%%%%%%%%%%%%%%%%%%%%%%%%%

